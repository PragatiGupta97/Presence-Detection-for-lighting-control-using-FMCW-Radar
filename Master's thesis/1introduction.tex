\chapter{Introduction}
\label{chapter:intro}
Presence detection has always been a hot topic in the industry. RGB cameras have proven to be quite accurate for detecting human presence. However, there are some limitations of using RGB in the indoor environments for industries like lighting industry. RGB camera hinders people’s privacy. People are usually not comfortable with the idea of a camera installed near them in order to track their activity. This is the main reason why RGB cameras cannot be used for indoor environments like offices.


The solution which is common now a days for lighting industry is thermal sensing using Passive Infrared sensors (PIR). PIRs are low-cost sensors and they have low power requirements and simple installation procedure. The problem with PIR is, it cannot detect stationary people and their output is bursty. Due to this reason usually, a timer is associated with the PIR data. The timer is reset to 0 whenever there is a movement in the room. This means if the PIR sensors don’t track any activity for the past 10 minutes (as an example for chosen timer range), only then the lights turn off. This often leads to power wastage. Even though, PIR sensors are a sustainable solution for lighting control, there is still a high scope for improvement.

This thesis is about lighting control using Radar Data, which tracks the human presence in the room. Unlike PIR, the range of the radar is huge, and one radar can offer as an alternate to numerous PIRs. Moreover, the radar data is fast and more reliable. The objective of this thesis is to measure true presence using radars, as compared to movement using PIRs. The objective will be achieved using Machine Learning classification models obtained from radar to make the machine understand about the human presence.


\section{Problem statement}




\section{Structure of the Thesis}
\label{section:structure} 


